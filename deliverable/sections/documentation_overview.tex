\section{Documentation Overview}

The repository collects implementation notes, compliance guidance, and operating procedures alongside the code. Documentation lives close to the artefacts it describes to encourage short feedback loops: developer onboarding material is surfaced in the root \texttt{README.md}, and deliverable-specific sections are assembled through this \LaTeX{} template. The \texttt{todo.md} backlog is treated as a living document that captures directives and tracks progress on required enhancements.

\subsection{Architecture Documentation}

High-level design rationale and cryptographic integration notes are maintained in this deliverable document. Section~3 (Avitabile Implementation) describes how the redactable blockchain core, smart contracts, and proof systems fit together, documenting the progression from protocol-level chameleon hash redaction to smart contract governance with zero-knowledge proofs. Section~4 (Implementation Details) provides technical depth on circuit design, SNARK integration, and backend architecture.

The top-level \texttt{README.md} complements these sections with an overview of core features, command entry points, and the relationship between on-chain commitments and off-chain IPFS artefacts. 

\textbf{Critical implementation status}: As documented in the Abstract and Appendix Section~A.3, the system infrastructure is code-complete but requires circuit recompilation to activate end-to-end functionality. The circuit source code defines 16 public signals, but compiled artifacts in \texttt{circuits/build/} contain only 1 signal from an earlier configuration. Appendix Section~A.1.3 provides complete recompilation procedures.

\subsection{Developer Documentation}

Developer-facing resources emphasise reproducibility. The \texttt{README.md} provides bootstrap commands, environment variables, and demo invocations. Module-specific docstrings (for example in \texttt{medical/MedicalRedactionEngine.py} and \texttt{adapters/snark.py}) document extension points, while the adapters include inline comments that justify non-obvious design decisions such as IPFS retry strategies. Configuration helpers under \texttt{adapters/config.py} and the generated badges in \texttt{badges/} surface build and coverage status for new contributors.

Key technical documentation includes:
\begin{itemize}
    \item \textbf{Circuit mapping}: \texttt{medical/circuit\_mapper.py} docstrings explain field element conversion and signal preparation for 16-signal proofs
    \item \textbf{SNARK integration}: \texttt{medical/my\_snark\_manager.py} documents proof generation workflow and nullifier extraction logic
    \item \textbf{Backend interfaces}: \texttt{medical/backends.py} describes EVM vs simulated backend switching
    \item \textbf{Test suite}: \texttt{tests/README.md} (if present) or test file docstrings explain unit vs integration test organization
\end{itemize}

\subsection{User-Facing Documentation}

User-oriented material focuses on demonstrators and workflow examples. The demo suite under \texttt{demo/} showcases standard redaction flows, IPFS integration, and zero-knowledge proof generation pathways. Key demos include:

\begin{itemize}
    \item \texttt{demo/avitabile\_redaction\_demo.py}: Multi-party approval governance workflow
    \item \texttt{demo/avitabile\_censored\_ipfs\_pipeline.py}: Censored data storage model
    \item \texttt{demo/avitabile\_consistency\_demo.py}: Consistency proof validation
    \item \texttt{demo/final\_demo.py}: Complete professor demonstration of Phase 1 and Phase 2 implementation
\end{itemize}

The medical use-case scripts demonstrate how synthetic datasets are generated and censored before being published via IPFS. 

\textbf{Operational readiness}: Demo scripts currently operate with 1-signal circuit artifacts. After circuit recompilation (see Appendix~A.1.3), demos will showcase full 16-signal proof verification with circuit-derived nullifiers and on-chain consistency validation. The deliverable, demo walkthroughs, and inline CLI help serve as the main references for stakeholders evaluating the prototype.
