\section{Documentation Overview}

The repository collects implementation notes, compliance guidance, and operating procedures alongside the code. Documentation lives close to the artefacts it describes to encourage short feedback loops: project-wide briefs reside under \texttt{docs/}, developer onboarding material is surfaced in the root \texttt{README.md}, and deliverable-specific sections are assembled through this \LaTeX{} template. The \texttt{todo.md} backlog is treated as a living document that captures directives from supervisors and tracks progress on required enhancements.

\subsection{Architecture Documentation}
High-level design rationale and cryptographic integration notes are maintained in \texttt{docs/ZK\_PROOF\_IMPLEMENTATION\_PLAN.md} and \texttt{docs/CONSISTENCY\_CIRCUIT\_INTEGRATION\_PLAN.md}. These documents describe how the redactable blockchain core, smart contracts, and proof systems fit together, and they are updated whenever major milestones land (e.g., the shift from simulated to real Groth16 proofs). The top-level \texttt{README.md} complements them with an overview of core features, command entry points, and the relationship between on-chain commitments and off-chain IPFS artefacts. Diagramming remains a TODO; the team keeps placeholders for future architecture figures as the contract orchestration layer stabilises.

\subsection{Developer Documentation}
Developer-facing resources emphasise reproducibility. The \texttt{README.md} provides bootstrap commands, environment variables, and demo invocations. Module-specific docstrings (for example in \texttt{medical/MedicalRedactionEngine.py} and \texttt{adapters/snark.py}) document extension points, while the adapters include inline comments that justify non-obvious design decisions such as IPFS retry strategies. Configuration helpers under \texttt{adapters/config.py} and the generated badges in \texttt{badges/} surface build and coverage status for new contributors. Outstanding documentation work includes formal coding standards and an explicit contribution guide.

\subsection{User-Facing Documentation}
User-oriented material focuses on demonstrators rather than polished manuals. The demo suite under \texttt{demo/} showcases standard redaction flows, IPFS integration, and zero-knowledge proof generation. The medical use-case notebooks and scripts document how synthetic datasets are generated and censored before being published via IPFS. A concise operator guide is planned once smart contract orchestration reaches feature parity with the simulator; in the meantime, the deliverable, demo walkthroughs, and inline CLI help serve as the main references for stakeholders evaluating the prototype.
